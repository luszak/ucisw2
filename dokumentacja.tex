\documentclass[a4paper,11pt]{article}
\usepackage{polski}
\usepackage[utf8]{inputenc}
\usepackage{latexsym}
\usepackage{graphicx} 
\usepackage{float}
\usepackage[margin=2.5cm]{geometry}
\usepackage{lscape}
\usepackage{listings}
\usepackage{color}
\usepackage{underscore}

\definecolor{mygreen}{rgb}{0,0.6,0}
\definecolor{mygray}{rgb}{0.5,0.5,0.5}
\definecolor{mymauve}{rgb}{0.58,0,0.82}

\lstset{ %
  backgroundcolor=\color{white},   % choose the background color; you must add \usepackage{color} or \usepackage{xcolor}; should come as last argument
  basicstyle=\footnotesize,        % the size of the fonts that are used for the code
  breakatwhitespace=false,         % sets if automatic breaks should only happen at whitespace
  breaklines=true,                 % sets automatic line breaking
  captionpos=b,                    % sets the caption-position to bottom
  commentstyle=\color{mygreen},    % comment style
  deletekeywords={...},            % if you want to delete keywords from the given language
  escapeinside={\%*}{*)},          % if you want to add LaTeX within your code
  extendedchars=true,              % lets you use non-ASCII characters; for 8-bits encodings only, does not work with UTF-8
  frame=single,	                   % adds a frame around the code
  keepspaces=true,                 % keeps spaces in text, useful for keeping indentation of code (possibly needs columns=flexible)
  keywordstyle=\color{blue},       % keyword style
  language=VHDL,                 % the language of the code
  morekeywords={*,...},            % if you want to add more keywords to the set
  numbers=left,                    % where to put the line-numbers; possible values are (none, left, right)
  numbersep=5pt,                   % how far the line-numbers are from the code
  numberstyle=\tiny\color{mygray}, % the style that is used for the line-numbers
  rulecolor=\color{black},         % if not set, the frame-color may be changed on line-breaks within not-black text (e.g. comments (green here))
  showspaces=false,                % show spaces everywhere adding particular underscores; it overrides 'showstringspaces'
  showstringspaces=false,          % underline spaces within strings only
  showtabs=false,                  % show tabs within strings adding particular underscores
  stepnumber=2,                    % the step between two line-numbers. If it's 1, each line will be numbered
  stringstyle=\color{mymauve},     % string literal style
  tabsize=2,	                   % sets default tabsize to 2 spaces
  title=\lstname,                  % show the filename of files included with \lstinputlisting; also try caption instead of title
  language=C++,
  literate={ą}{{\k{a}}}1
             {Ą}{{\k{A}}}1
             {ę}{{\k{e}}}1
             {Ę}{{\k{E}}}1
             {ó}{{\'o}}1
             {Ó}{{\'O}}1
             {ś}{{\'s}}1
             {Ś}{{\'S}}1
             {ł}{{\l{}}}1
             {Ł}{{\L{}}}1
             {ż}{{\.z}}1
             {Ż}{{\.Z}}1
             {ź}{{\'z}}1
             {Ź}{{\'Z}}1
             {ć}{{\'c}}1
             {Ć}{{\'C}}1
             {ń}{{\'n}}1
             {Ń}{{\'N}}1
}


\begin{document}

\begin{titlepage}

\newcommand{\HRule}{\rule{\linewidth}{0.5mm}}
\center
 
\textsc{\LARGE Politechnika Wrocławska}\\[1.5cm] 
\textsc{\Large Układy cyfrowe i systemy wbudowane}\\[0.5cm] %TU

\HRule \\[0.5cm]
{ \huge \bfseries Dokumentacja projektu \\ Organy z możliwością zapisywania i odtarzania melodii.}\\[0.5cm] %TU
\HRule \\[1.6cm]
 
%\begin{minipage}{0.4\textwidth}
\begin{flushleft} \large

\emph{Autorzy:}\\
Łukasz  \textsc{Bieszczad}\\ %TU
Krzysztof  \textsc{Buczak}\\ %TU

\end{flushleft}
%\end{minipage}

%\begin{minipage}{0.4\textwidth}
\begin{flushright} \large

\emph{Prowadzący:} \\
dr inż. Jarosław \textsc{Sugier} %TU

\end{flushright}
%\end{minipage}\\[4cm]

\vfill
{\large 18 maja 2018}\\[3cm] %TU %TU


\end{titlepage}

\section{Wprowadzenie}
\subsection{Cel i zakres}
Celem projektu było stworzenie jednooktawowego "instrumentu" klawiszowego, obsługiwanego za pomocą klawiatury PS/2. Wciskanie poszczególnych klawiszy miało powodować odtwarzanie dźwięków przez podpięty do pinów głośniczek. Dodatkowo częścią zadania było także zaimplementowanie funkcjonalności nagrywania melodii (zapis do pamięci ROM) i odtwarzania nagranego materiału, a także wykorzystanie wyświetlacza LCD do pokazania stanu nagrywania i diody LED do przekazania informacji o trwającym właśnie nagrywaniu.

\subsection{Zagadnienia teoretyczne}

\subsection{Sprzęt}
Językiem projektu był język opisu sprzętu VHDL. Stanowisko laboratoryjne/projektowe zostało wyposażone w układ Spartan-3E oraz komputer z oprogramowaniem Xilinx ISE, pozwalającym kompilować kod VHDL pod dostarczony sprzęt, a także wykonywać symulacje testujące działanie systemu. Wykorzystaliśmy także port PS/2 (klawiatura symulująca keyboard), piny do podłączenia głośnika, ekran LCD wyświetlający stan nagrywania (pozostały czas) i układ pamięci ROM do przechowywania nagranych melodii.

\section{Projekt}
\subsection{Schemat i hierarchia projektu}
\subsection{Moduły}

\section{Implementacja}
\subsection{Zasoby}
\subsection{"User manual" urządzenia}

\section{Podsumowanie}
Zadanie udało się zrealizować w całości. Instrument jest w pełni działający, a ze względu na jasny podział na moduły można bez trudu dopisywać do niego kolejne funkcjonalności. Także sama wartość "merytoryczna" keyboarda nie pozostawia wiele do życzenia, ponieważ faktycznie pokrywa on całą oktawę, a wysokości dźwięków różnią się od siebie dokładnie tak jak w prawdziwym instrumencie, dzięki czemu mając nuty do utworu muzycznego możemy go zagrać.

\section{Literatura}

\end{document}

